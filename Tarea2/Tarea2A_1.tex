\documentclass[11pt,a4paper]{article}
\usepackage[utf8]{inputenc}
\usepackage[spanish]{babel}
\usepackage{amsmath}
\usepackage{amsfonts}
\usepackage{amssymb}
\usepackage{graphicx}
\usepackage{booktabs} % Para tablas de mejor calidad
\usepackage{geometry}
\geometry{a4paper, margin=1in}
\usepackage{float} % Para mejor control de figuras [H]

\title{Reporte Ejecutivo: Análisis Bootstrap y Monte Carlo \\ para la Estimación de $\tau(\theta) = P(X=0)$ en una Distribución Poisson}
\author{Omar Flores \\ (Basado en el script R proporcionado)}
\date{\today}

\begin{document}
\maketitle
\begin{abstract}
Este reporte presenta los resultados de análisis estadísticos realizados para estimar el parámetro $\tau(\theta) = e^{-\theta} = P(X=0)$ de una distribución Poisson, utilizando el estimador UMVUE $\hat{\tau} = \left(\frac{n-1}{n}\right)^{\sum X_i}$. Se aplicaron métodos de Monte Carlo y bootstrap no paramétrico según lo especificado, y se evaluó el desempeño de los intervalos de confianza bootstrap. Todos los análisis se implementaron en R, utilizando `set.seed(1234)` para reproducibilidad. Las cifras y tablas aquí presentadas se derivan de la ejecución de dicho script.
\end{abstract}

\hrulefill

\hrulefill
\vspace{1em}

\section{Introducción}
El objetivo principal es la estimación del parámetro $\tau(\theta) = e^{-\theta}$, que representa la probabilidad de observar un cero en una variable aleatoria $X$ que sigue una distribución de Poisson con parámetro $\theta$. El estimador utilizado es $\hat{\tau} = \left(\frac{n-1}{n}\right)^{\sum_{i=1}^n X_i}$, el cual se indica es el estimador insesgado de mínima varianza uniforme (UMVUE) de $\tau(\theta)$.

Este reporte se divide en tres partes principales:
\begin{enumerate}
    \item \textbf{Simulación Monte Carlo (MC):} Aproximación del valor esperado $E(\hat{\tau})$ y la varianza $V(\hat{\tau})$ cuando $\theta$ es conocido.
    \item \textbf{Bootstrap No Paramétrico:} Estimación de $\tau(\theta)$, $V(\hat{\tau})$ y construcción de un intervalo de confianza para $\tau(\theta)$ a partir de una muestra observada específica, sin asumir conocimiento de $\theta$.
    \item \textbf{Estudio de Simulación de Cobertura:} Evaluación de la probabilidad de cobertura de los intervalos de confianza bootstrap no paramétricos.
\end{enumerate}
Para todos los análisis inferenciales no especificados, se utiliza un nivel de significancia $\alpha = 0.05$.

\section{Parte a: Simulación Monte Carlo (Distribución Conocida)}
En esta sección, se utiliza el método Monte Carlo para aproximar el valor esperado y la varianza del estimador $\hat{\tau}$. Se asume que los datos provienen de una distribución Poisson($\theta$) con $\theta = 1.3$, y se generan muestras de tamaño $n=20$. Se realizaron $B = 10,000$ réplicas Monte Carlo.

\subsection{Metodología}
Para cada una de las $B=10,000$ réplicas:
\begin{enumerate}
    \item Se generó una muestra aleatoria $X_1, \dots, X_{20}$ de una distribución Poisson($\theta=1.3$).
    \item Se calculó $\hat{\tau}_b = \left(\frac{20-1}{20}\right)^{\sum X_i}$ para la muestra $b$.
\end{enumerate}
Luego, $E(\hat{\tau})$ se aproximó por $\frac{1}{B}\sum_{b=1}^B \hat{\tau}_b$ y $V(\hat{\tau})$ por $\frac{1}{B-1}\sum_{b=1}^B (\hat{\tau}_b - \bar{\hat{\tau}})^2$.

\subsection{Resultados}
El valor verdadero de $\tau(\theta)$ para $\theta=1.3$ es $\tau(1.3) = e^{-1.3}$.
Los resultados obtenidos de la simulación Monte Carlo se resumen en la Tabla \ref{tab:mc_results}.

\begin{table}[H]
    \centering
    \caption{Resultados de la Simulación Monte Carlo para $\hat{\tau}$ ($\theta=1.3, n=20, B=10,000$)}
    \label{tab:mc_results}
    \begin{tabular}{lc}
        \toprule
        Métrica & Valor \\
        \midrule
        Valor verdadero $\tau(1.3) = e^{-1.3}$ & [VALOR_TAU_VERDADERO_A] \\
        Estimación MC de $E(\hat{\tau})$ & [VALOR_E_TAU_HAT_MC] \\
        Estimación MC de $V(\hat{\tau})$ & [VALOR_V_TAU_HAT_MC] \\
        \bottomrule
    \end{tabular}
    \floatfoot{Nota: Los valores numéricos deben ser completados ejecutando el script R proporcionado (`[PLACEHOLDER_VALUE]`).}
\end{table}

El histograma de las $10,000$ estimaciones $\hat{\tau}_b$ obtenidas se muestra en la Figura \ref{fig:mc_hist_a}.

\begin{figure}[H]
    \centering
    \includegraphics[width=0.8\textwidth]{histograma_mc_tau_parte_a.pdf}
    \caption{Histograma de las estimaciones $\hat{\tau}$ obtenidas mediante simulación Monte Carlo. La línea roja discontinua indica el valor verdadero $\tau(1.3)$ y la línea azul punteada indica la media de las estimaciones MC.}
    \label{fig:mc_hist_a}
    \floatfoot{Nota: Esta figura es generada por el script R como `histograma_mc_tau_parte_a.pdf`.}
\end{figure}



\section{Parte b: Bootstrap No Paramétrico (Muestra Observada)}
En esta sección, se utiliza el método bootstrap no paramétrico para analizar el estimador $\hat{\tau}$ a partir de una única muestra observada, sin suponer conocimiento previo de $\theta$ o la forma exacta de la distribución (aunque el estimador $\hat{\tau}$ se deriva bajo el supuesto Poisson).

\subsection{Datos y Metodología}
La muestra observada es $X = (1, 2, 0, 0, 1, 1, 0, 1, 0, 0, 1, 1, 0, 1, 1, 0, 0, 1, 1, 0)$, con $n=20$.
Se generaron $B=10,000$ muestras bootstrap:
\begin{enumerate}
    \item Se calculó $\hat{\tau}_{obs}$ a partir de la muestra original.
    \item Para cada una de las $B=10,000$ réplicas bootstrap:
    \begin{enumerate}
        \item Se generó una muestra bootstrap $X_1^*, \dots, X_{20}^*$ muestreando con reemplazo de la muestra observada $X$.
        \item Se calculó $\hat{\tau}_b^*$ para la muestra bootstrap $b$.
    \end{enumerate}
    \item La varianza de $\hat{\tau}$ se estimó como la varianza de las $\hat{\tau}_b^*$.
    \item Se construyó un intervalo de confianza (IC) del 95\% para $\tau(\theta)$ utilizando el método de los percentiles sobre las $\hat{\tau}_b^*$.
\end{enumerate}

\subsection{Resultados}
Los resultados del análisis bootstrap no paramétrico se resumen en la Tabla \ref{tab:boot_results_b}.

\begin{table}[H]
    \centering
    \caption{Resultados del Bootstrap No Paramétrico para $\hat{\tau}$ (Muestra observada, $n=20, B=10,000$)}
    \label{tab:boot_results_b}
    \begin{tabular}{lc}
        \toprule
        Métrica & Valor \\
        \midrule
        Estimación $\hat{\tau}_{obs}$ (de la muestra original) & [VALOR_TAU_HAT_OBS_B] \\
        Estimación Bootstrap de $V(\hat{\tau})$ & [VALOR_V_TAU_HAT_BOOT_B] \\
        IC del 95\% para $\tau(\theta)$ (percentiles) & ([LIM_INF_IC_B], [LIM_SUP_IC_B]) \\
        \bottomrule
    \end{tabular}
    \floatfoot{Nota: Los valores numéricos deben ser completados ejecutando el script R (`[PLACEHOLDER_VALUE]`).}
\end{table}

El histograma de las $10,000$ estimaciones bootstrap $\hat{\tau}_b^*$ se muestra en la Figura \ref{fig:boot_hist_b}.

\begin{figure}[H]
    \centering
    \includegraphics[width=0.8\textwidth]{histograma_bootstrap_tau_parte_b.pdf}
    \caption{Histograma de las estimaciones bootstrap $\hat{\tau}^*$. La línea azul discontinua indica $\hat{\tau}_{obs}$ y las líneas púrpuras punteadas los límites del IC del 95\%.}
    \label{fig:boot_hist_b}
    \floatfoot{Nota: Esta figura es generada por el script R como `histograma_bootstrap_tau_parte_b.pdf`.}
\end{figure}



\section{Parte c: Estudio de Simulación para Cobertura del IC Bootstrap}
Se realizó un estudio de simulación para evaluar el desempeño (probabilidad de cobertura) de los intervalos de confianza del 95\% obtenidos mediante el método bootstrap no paramétrico (percentiles) para $\hat{\tau}$.

\subsection{Metodología}
Se repitió el siguiente procedimiento $M=1,000$ veces:
\begin{enumerate}
    \item Se generó una muestra aleatoria de tamaño $n=20$ de una distribución Poisson($\theta=1.3$). El verdadero valor del parámetro de interés es $\tau(1.3) = e^{-1.3}$.
    \item Con esta muestra generada, se construyó un intervalo de confianza del 95\% para $\tau(\theta)$ usando el método bootstrap no paramétrico con $B=10,000$ réplicas (como en la Parte b).
    \item Se verificó si el intervalo de confianza resultante contenía el valor verdadero $\tau(1.3)$.
\end{enumerate}
La probabilidad de cobertura se estimó como la proporción de los $M$ intervalos que contenían el valor verdadero.

\subsection{Resultados}
Los resultados del estudio de simulación de cobertura se presentan en la Tabla \ref{tab:coverage_results_c}.

\begin{table}[H]
    \centering
    \caption{Resultados del Estudio de Simulación de Cobertura del IC Bootstrap ($M=1000$ sims, $\theta=1.3, n=20, B=10000$ por IC)}
    \label{tab:coverage_results_c}
    \begin{tabular}{lc}
        \toprule
        Métrica & Valor \\
        \midrule
        Nivel de Confianza Nominal & 0.95 \\
        Probabilidad de Cobertura Observada & [VALOR_COBERTURA_OBS_C] \\
        Número de ICs que cubrieron $\tau(1.3)$ (de $M=1000$) & [NUM_COBERTURAS_C] \\
        $p$-valor (Prueba Binomial $H_0$: Cobertura = 0.95) & [P_VALOR_BINOM_TEST_C] \\
        \bottomrule
    \end{tabular}
    \floatfoot{Nota: Los valores numéricos deben ser completados ejecutando el script R (`[PLACEHOLDER_VALUE]`).}
\end{table}




\vspace{2em}
\hrulefill

\end{document}
